# Graph Theory

# General properties

### Degrees

\begin{theorem} \label{sumIndegreesIsNrEdges}
    In a directed graph $\sum_{V_G} \indeg(v_n) = |E_G|$
\end{theorem}

\begin{proof}
    \induction{$ \sum_{V_G} \indeg(v_n) = |E_G| $ \\}
    {$\sum_{V_G} \indeg(v_n) = |E_G|$ where $E_G = \{(v_1, v_2)\}$ \\}
    {This is tirvially true.}
    {
    \begin{itemize}
        \item Let $\sum_{V_G} \indeg(v_n) = |E_G|$ 
        \item Let $V_{G'} = V_G \cup \{v_{N+1}\}$
        \item Let $E_{G'} = E_G \cup \{(x, y) \in V_G \times \{v_{N+1}\} \cup \{v_{N+1}\} \times V_G \}$
    \end{itemize}
    }
    {
     For the left side it holds that: 
                $$ \sum_{V_{G'}} \indeg(v_n) = \sum_{V_G} \indeg(v_n) + \outdeg(v_{N+1}) + \indeg(v_{N+1}) $$
                On the other hand on the right side it holds that: 
                $$ |E_{G'}| = |E_G| + \outdeg(v_{N+1}) + \indeg(v_{N+1}) $$
                So the equation reduces to $0=0$, which is true.
    }
\end{proof}


\begin{lemma} \label{indegIsOutdeg}
    $\sum_{V_G} \indeg(v_n) = \sum_{V_G} \outdeg(v_n)$
\end{lemma}

\begin{lemma}
    In an undirected graph we have $\sum_{V_G} \degr(v_n) = 2|E_G|$
\end{lemma}

# Walks and paths

\begin{definition}
    A walk is any sequence of vertices that are connected by an edge.
\end{definition}

\begin{definition}
    A path is a walk where no vertex appears twice.
\end{definition}

\begin{theorem}
    The shortest walk between any two vertices in a graph is a path.
\end{theorem}

\begin{proof}
    \subprf{By contradiction. Suppose that the vertex $x$ appears twice in the shortest walk.}{this is a contradiction.}{
        Since $x$ appears twice, the proposed walk must be of the form $f-x-g-x-h$, where $g$ is a sequence of $0$ or more connected vertices. Then this walk can be shortened to $f-x-h$.
    }
\end{proof}

\begin{theorem}
    The longest path in a graph has length $ | V_G | - 1 $.
\end{theorem}
This is too trivial for a full-fledged proof: there are no repetitions in a path, so it can only have as many steps as it has vertices.


### Relations and adjacency matrices

Let $R \circ Q$ be a composition of two relations. The number of paths of length exactly $n$ between two vertices $x$ and $y$, written as $\paths_n(x, z)$, can be expressed  as 
 $$ \paths_n(x,z) = | \{ y | xRy \land yQz \} | = \sum_{y \in Y} xRy \cdot yQz$$


If $R_{AM}$ and $Q_{AM}$ are the adjacency matrix representations of the above relations, then: 

$$  R_{AM} \cdot Q_{AM} [n,m] = \sum_{y \in Y} R_{AM} [x_n, y] \cdot Q_{AM} [y, z_m] $$

$$ = \sum_{y \in Y} xRy \cdot yQz = \paths_n(x,z) $$

When $R=Q$, it follows easily that:

\begin{theorem}
    $R_{AM}^2 [n,m]$ is the number of paths  that go from $v_n$ to $v_m$ in exactly $2$ steps.
\end{theorem}

\begin{theorem} \label{all_possible_paths}
    $R_{AM}$ explains whether or not two vertices $v$ and $u$ are connected. All possible paths are listed in $R^* = R_{AM} \cup R_{AM}^2 \cup R_{AM}^3 \cup ... \cup R_{AM}^{|V_G|-1}$.
    We can find the length of the shortest path between two vertices $u$ and $v$ by
\end{theorem}



# Planar graphs

The following theorems all deal with planar, connected graphs, and build up to Eulers theorem. Well use $N$ for the number of nodes, $E$ for the number of edges, and $L$ for the number of loops.

\begin{theorem}
    Adding a edge to a planar, connected graph means adding one loop: $ \Delta N = 0 \land \Delta E = 1 \then \Delta L = 1$
\end{theorem}

\begin{theorem}
    Adding x edges means adding x loops: $ \Delta N = 0 \land \Delta E = x \then \Delta L = x $
\end{theorem}

\begin{proof}
    \subprf{}{$\Delta N = 0 \land \Delta E = x \then \Delta L = x$}{
        By induction on $\Delta E$. \\
        \subprf{Base case: $\Delta E = 1$}{$\Delta L = 1$}{
            By theorem 11.
        }
        \subprf{Let $\Delta N = 0 \land \Delta E = x \then \Delta L = x$.\\}
            {$\Delta N = 0 \land \Delta E = x+1 \then \Delta L = x+1$}{
            Consider the graph from the induction hypothesis, then apply theorem 12.
        }
    }
\end{proof}

\begin{theorem}
    When we add a new node to a graph and connect it to the graph using one or more edges, then the number of edges and loops behaves as such: $ \Delta N = 1 \then \Delta L = \Delta E - 1 $
\end{theorem}

\begin{proof}
    \subprf{}{$\Delta N = 1 \then \Delta L = \Delta E - 1$}{
        Let $ \Delta N = 1 $. \\
        By induction on $ \Delta E $. \\
        \subprf{Base case: $\Delta E = 1 $}{$ \Delta L = 0 $}{
            Graphically trivial.
        }
        \subprf{Induction step: $\Delta L = \Delta E - 1$}{$\Delta L' = \Delta E' - 1$}{
            $\Delta E ' = \Delta E + 1$ \\
            Using theorem 12, that means:
            $\Delta L' = \Delta L +1$ \\
            Thus: 
            $\Delta L + 1 = \Delta E + 1 - 1 $ \\
            Which reduces to: 
            $\Delta L = \Delta E - 1$, which is true by the induction hypothesis.
        }
    }
\end{proof}

\begin{theorem}
    Eulers theorem: $N + L = E + 1$
\end{theorem}

\begin{proof}
    \subprf{}{$N + L = E + 1$}{
        By induction on $N$. \\
        \subprf{Base case: $N=1$ or $N=2$}{$N + L = E + 1$}{
            If $N=1$: $1 + 0 = 0 + 1$ \\
            If $N=2$: $2 + 0 = 1 + 1$
        }
        \subprf{Induction step: Let $N + L = E + 1$.\\}{$N+1 + L' = E' + 1$}{
            Using theorem 13: $\Delta L' = \Delta E - 1$ \\
            Thus:
            $N + 1 + L + \Delta L = E + \Delta E + 1$ \\
            $N + L = E + 1$, which is true  by the induction hypothesis.
        }
    }
\end{proof}



### Trees


Here is a fun little proof.

\begin{proof} In a tree, the mean number of children ($mcc$) is always equal to $1 - 1/N$.
    
    \subprf{}{$mcc = 1 - \frac{1}{N}$}{
        With what we have so far, this is almost trivial to prove. 
        The mean child count can be written as 
        $$ mcc = \frac{1}{N} \sum_N \outdeg{v_n} $$
        Using theorem \ref{sumIndegreesIsNrEdges} and lemma \ref{indegIsOutdeg}, this becomes
        $$ mcc = \frac{1}{N} E $$
        Eulers formula in a tree becomes $N = E + 1$, since there are no loops in trees. Using this: 
        $$ mcc = 1 - \frac{1}{N} $$
    }

\end{proof}

When training a binary classifier, we usually need a measure of defining which of several trees is a good representation of the data. The number of nodes, however, cannot be used as a measure, because it is always going to be $2n - 1$.
\begin{proof} In a binary tree, the number of leaves $n$ equals $\frac{N+1}{2}$, regardless of the architecture of the tree.

    \subprf{}{$n = \frac{N+1}{2}$}{
        We shall use the fact that in any tree, $\sum_{V_G} \indeg(v_n) = \sum_{V_G} \outdeg(v_n)$. \\

        Consider the leaves of the tree. They all have $\indeg(v_n) = 1$ and $\outdeg(v_n) = 0$. Thus $\sum_{\text{leaves}} \indeg(v_n) = n$ and $\sum_{\text{leaves}} \outdeg(v_n) = 0$. \\
        Now consider the nodes (there are $N-n$ of them). They all have $\outdeg(v_n) = 2$ and $\indeg(v_n) = 1$, except for the root-node. Thus $\sum_{\text{nodes}} \indeg(v_n) = N - n - 1$ and $\sum_{\text{nodes}} \outdeg(v_n) = 2(N - n)$. \\
        Equating the in- and out-degrees, we obtain:
        $$ n + N - n - 1 = 0 + 2(N - n) $$
        $$ n = \frac{N+1}{2} $$
    }

\end{proof}


### Triangulation

\begin{proof} A \emph{n-gon}, a polygon with n nodes, can be drawn with $n-2$ triangles.
The proof follows from induction on $n$, starting at $n=3$.
\end{proof}

\begin{proof} When drawing a n-gon ($n \geq 3$) using triangles, there are $n-3$ shared edges.
\begin{itemize}
    \item The n-gon has $n$ nodes, $n$ edges and one loop
    \item The $n-2$ triangles have in total $n$ nodes, $n-2$ loops and (via Eulers formula) $2n-3$ edges.
    \item The outer edges, that is, all the edges of the n-gon, are not shared. Thus, there are $e_{triangles} - e_{n-gon} = n-3$ shared edges.
\end{itemize}
\end{proof}



\subsection{Stable marriage and Gayle-Shapely}
In dating, the people doing the active flirting end up with partners higher up their preference-list than the people being flirted with.
\begin{lstlisting}[language=python]
    class Person:
    def __init__(self, name):
        self.name = name
        self.prefs = []

    def setPreferences(self, prefs):
        self.prefs = prefs

    def __repr__(self):
        return self.name


class Woman(Person):
    def selectBest(self, suitors):
        for p in self.prefs:
            if p in suitors:
                print(f"{self} selected {p}")
                return p


class Man(Person):
    def getFavorite(self):
        print(f"{self} fancies {self.prefs[0]}")
        return self.prefs[0]

    def rejectedBy(self, w):
        print(f"{w} rejected {self}")
        self.prefs.remove(w)


rachel = Woman('Rachel')
phoebe = Woman('Phoebe')
monica = Woman('Monica')

ross = Man('Ross')
chandler = Man('Chandler')
joey = Man('Joey')

women = [rachel, phoebe, monica]
men = [ross, chandler, joey]

rachel.setPreferences([joey, ross, chandler])
phoebe.setPreferences([ross, chandler, joey])
monica.setPreferences([joey, chandler, ross])

ross.setPreferences([rachel, phoebe, monica])
chandler.setPreferences([rachel, monica, phoebe])
joey.setPreferences([phoebe, rachel, monica])


## Gayle - Shapely

flirting = {w: [] for w in women}

def isStable(flirting):
    for w in flirting:
        suitors = flirting[w]
        if len(suitors) != 1:
            return False
    return True


while not isStable(flirting):
    print("--- lets go, flirt! ----")

    # men move to their most desired woman
    for m in men:
        w = m.getFavorite()
        flirting[w].append(m)

    # women reject all but best suitor
    for w in women:
        suitors = flirting[w]
        fav = w.selectBest(suitors)
        for m in suitors:
            if m is not fav:
                m.rejectedBy(w)
        if fav:
            flirting[w] = [fav]


print(" ---- couples have formed: ---- ")
for w in flirting:
    m = flirting[w][0]
    print(f"{w} + {m}")
\end{lstlisting}


### Path finding
We have a graph of connections with a cost associated to each connection.
The goal is to go from node a to node b as cheaply as possible.

A first approach might be depth first. But depth first on a plain, empty field will snake left and right along the bottom 
for a long time before it finally hits a point b just a few boxes above a.

Then we're better off with breadth first search. But BFS spreads out blindly in all directions.

The next better thing is Dijkstra's algorithm. It's BFS but with a preference for cheap paths.
On a road net, Dijkstra would investigate highways before it considers dirt roads with trees across them.
All the points considered in BFS are ranked in a priority queue.
The queue contains the candidate point, the cheapest path to the point up to now, and the cheapest price up to now.
We pick the cheapest point first. If our path to that current cheapest point is cheaper than what it was up to now, that path is updated.

Notably, Dijkstra has no sense of direction.
If a dirt-track leads almost directly north to b, but a highway goes south, Dijkstra will investigate the highway first.
This can be accounted for by adding a distance-heuristic to the price-calculation.
This is the A* algorithm.

\begin{lstlisting}
type Node = char;
type Connection = [Node, Node];

class Graph {
    nodes: Node[] = [];
    connections: Connection[] = [];
}

class QEntry {
    node: Node;
    path: Node[];
    price: number;
}

class Queue {
    entries: QEntry[];

    public getNext(): QEntry {
        return this.entries.pop();
    }

    public add(entry: QEntry): void {
        for (let i = this.entries.length; i > 0; i--) {
            const candidate = this.entries[i];
            if (candidate.price < entry.price) {
                this.entries.insert(i, entry);
            }
        }
    }
}

function dijkstra(graph: Graph, queue: Queue, b: Node): Node[] {
    const a = queue.getNext();
    for (const {candidateNode, candidatePrice } in graph.getNeighbors(a.node)) {
        if (candidateNode === b) {
            return a.path + [b];
        }
        const newEntry: QEntry = { candidateNode, a.path + [a.node], a.price + candidatePrice };
        if (queue.get(candidateNode).price > newEntry.price) {
            queue.add(newEntry);
        }
    }
    return dijkstra(graph, queue, b);
}
\end{lstlisting}

There are also public APIs for path-finding on actual streets. \href{https://www.graphhopper.com/}{Graphhopper} is one example.
